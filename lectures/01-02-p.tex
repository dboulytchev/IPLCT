\documentclass[aspectratio=43]{beamer}
\usepackage[utf8]{inputenc}
\usepackage[english,russian]{babel}
\usepackage[T2A]{fontenc}


\sloppy

\newcommand{\lama}{$\lambda\kern -.1667em\lower -.5ex\hbox{$a$}\kern -.1000em\lower .2ex\hbox{$\mathcal M$}\kern -.1000em\lower -.5ex\hbox{$a$}$\xspace}

\mode<presentation>{
  \usetheme{IPLC}
}

\begin{document}

\begin{frame}[fragile]{Объекты и представления}
x
\end{frame}

\begin{frame}[fragile]{Языки программирования и семантика}
  \[
  \begin{array}{ccl}
    \mathcal L & - & \mbox{язык программирования}\\
               &   & \mbox{(набор представлений программ)}\\[2mm]
    \mathcal D & - & \mbox{семантический домен}\\[2mm]
    \Sem{\bullet}{L} : \mathcal L \mapsto \mathcal D & - & \mbox{семантика}
  \end{array}
  \]
  
  Примеры
  
  \begin{itemize}
  \item Глупый:
    \[
    \begin{array}{ccc}
      \mathcal D = \mathbb N &, & \sembr{\bullet}\equiv 42
    \end{array}
    \]
  \item Умный:
    \[
    \begin{array}{ccc}
      \mathcal D = \mathbb N \to \mathbb N&, & \sembr{\mathcal L} \supseteq \{\mbox{частично-рекурсивные функции}\}
    \end{array}
    \]    
  \end{itemize}
\end{frame}

\begin{frame}[fragile]{Метапрограммирование}
x
\end{frame}

\begin{frame}[fragile]{Интерпретаторы}
  \[
  \begin{array}{c}
    \Int{L}{M}\in \mathcal M\\
    \Sem{\Int{L}{M}}{M}\,(p_{\mathcal L},\,x)=\Sem{p_{\mathcal L}}{L}\,(x)
  \end{array}
  \]
\end{frame}


\end{document}
