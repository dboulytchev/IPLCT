\chapter{Semantics and Interpreters}

In this section we set the foundations for formal semantics which will be used in the rest of the course. We also discuss the
relation between programs and their representation in a concrete data domain, introduce the notion of interpreter and
consider some sample languages, their semantics and techniques for interpreter implementations.

\section{Languages and Semantics}

We consider programming language $\mathcal L$ as a (countable) set of programs

\[
\mathcal{L}=\{\primi{p}_1,\,\primi{p}_2,\dots\}
\]

To give a \emph{semantics} for the language $\mathcal L$ is to specify two objects:

\begin{itemize}
\item a \emph{semantic domain} $\mathcal D$;
\item a \emph{total} mapping $\sembr{\bullet}^\ph_{\mathcal L} : \mathcal L \to \mathcal D$.
\end{itemize}

Thus, for a program

\[
\primi{p}\in\mathcal L
\]

its semantics is just a

\[
\sembr{\primi{p}}^\ph_{\mathcal L}\in\mathcal D
\]

When the language is easily deducible from the context we will omit the subscript and write simply $\sembr{\bullet}$.

By claiming the totality of $\sembr{\bullet}$ we
make sure that any program has some semantics. The nature of semantic domain $\mathcal D$ essentially defines the nature of $\sembr{\bullet}_{\mathcal L}$;
for example, we can set

\[
\mathcal D=\{\Cat\}
\]

and (the only choice)

\[
\sembr{\primi{p}}=\Cat
\]

for every $\primi{p}\in\mathcal L$. We admit that this particular
semantics might not be very useful (this, however, depends on the nature of $\mathcal L$), but the important observations are that

\begin{itemize}
\item the choice of $\mathcal D$ has to be made;
\item there might be (and, \emph{as a rule}, are) multiple semantics for a given language.
\end{itemize}

These various semantics for a language may reflect its various properties (and we will see some of those in a little while); however, as a rule, one particular
semantics is chosen as the ``standard'' one.

\section{Data Domain}

If we speak of a general-purpose programming language and interested in its execution semantics then, probably, we may consider choosing the semantic domain
to be the set of all \emph{partially-recursive} functions and $\sembr{p}$ to be the function $p$ evaluates. This choice, however, would be too high-level and
abstract for our purposes.

To be more concrete, we first choose a \emph{data domain} $\mathfrak D$ to be the set of all reasonable data values the programs can take as inputs and
produce as outputs; then the semantic domain for executable semantics would be

\[
\mathcal{D}:\mathfrak{D}\to\mathfrak{D}
\]

Thus, executable semantics maps programs to data-processing functions.

In order to make further progress we stipulate the following properties of the data domain. First, we require its \emph{closedness under product}:

\[
\mathfrak{D}\times\mathfrak{D}\subset\mathfrak{D}
\]

In other words, $\mathfrak{D}$ contains all pairs, triples, etc. of its values.

The next requirement follows from out intention to make \emph{metaprogramming} possible. In short, the idea behind metaprogramming is
to use \emph{programs} as \emph{data}. Indeed, all programming tools follow this approach: a compiler takes a program as input and
returns (another) program as output, etc. This can be done only if programs can be \emph{encoded} somehow in the data domain. We
consider some concrete encodings, using \lama data domain later; for now we assume that for arbitrary programming language $\mathcal{L}$
the set $\mathfrak{D}$ contains the \emph{representations}
of all programs in $\mathcal{L}$:

\[
\forall \mathcal{L}\; .\; \mathcal{L}\subset\mathfrak{D}
\]

We agreed above to consider programming languages as sets of programs; we could, equivalently, consider them as sets of \emph{representations} of
programs in some universal data domain. Since the correspondence between programs and their representations is one-to-one this convention would
not hinder any follow-up reasonings. From now on we will not distinguish programs (abstract objects) from their representations (concrete objects)
in $\mathfrak{D}$. Note, there can be multiple representations within one data domain, but all of them are ``equivalent'' in the sense that
each pair of them admits an unambiguous conversions in both directions.

\section{Semantic Properties}

Let us have a language $\mathcal{L}$ and its \emph{two} semantics

\[
\begin{array}{rcl}
    \sembr{\bullet}^\ph & : & \mathcal{L} \to \mathcal{D}\\
  \sembr{\bullet}^\prime & : & \mathcal{L} \to \mathcal{D}
\end{array}
\]

with the \emph{same} semantic domain. We say that these semantics are \emph{equivalent} if and only if for arbitrary program $\primi{p}\in\mathcal{L}$

\[
\sembr{\primi{p}}^\ph=\sembr{\primi{p}}^\prime
\]

In other words, equivalent semantics assign to each program in the language the same element of semantic domain. A question might arise why would we
need two equivalent semantics; wouldn't the single one be sufficient? The answer is that there are multiple ways of
describing semantics, and these different ways have different properties which make them preferable in different settings. Sometimes it is desirable to
reformulate the semantics in different terms. By proving the equivalence between the two we can justify that we still deal with the language with the same
semantic properties.

Note, when the semantic domain is domain of functions $\mathfrak{D}\to\mathfrak{D}$, the equation above denotes the equality of functions: for
arbitrary $x, y\in\mathfrak{D}$

\[
\sembr{\primi{p}}^\prime\,x=y \Leftrightarrow \sembr{\primi{p}}\,x=y
\]

In other words, in both semantics $\primi{p}$ is defined on exactly the same inputs and for each of these inputs it provides the same outputs.

Another important property is \emph{equivalence of programs} within the same semantics. We say that $\primi{p}_1$ is (semantically) equivalent to $\primi{p}_2$ if
and only if

\[
\sembr{\primi{p}_1}=\sembr{\primi{p}_2}
\]

Thus, equivalent programs have the same semantics. The equivalence of \emph{different} programs within the same semantics plays a crucial
role in justifying the correctness of program transformations. Let us have some transformation of programs into programs:

\[
f : \mathcal{L}\to\mathcal{L}
\]

We say that $f$ is \emph{semantically correct} if and only if for all programs $\primi{p}$

\[
\sembr{f\,(\primi{p})}=\sembr{\primi{p}}
\]

Thus, equivalent transformations do not change the semantics of programs; they can, however, change other their properties.

When the semantic domain is domain of functions, the equation above, again, denotes the equality of functions; additionally,
the following important notion can be introduced in this case. We say that $f$ is \emph{partially correct} if and only if
for all programs $\primi{p}$ and all $x, y\in\mathfrak{D}$

\[
\sembr{\primi{p}}\,x=y\Rightarrow\sembr{f\,(\primi{p})}\,x=y
\]

The difference between correctness and partial correctness is that in the former case the programs are defined for exactly the same inputs,
while in the latter the transformed program can be defined even if the original one in not. To some extent this is how
optimizing transformations work, as we've seen in the previous chapter.

Finally, there can be transformations between \emph{different} languages with the \emph{same} semantic domain. Let us have two languages $\mathcal{L}$ and $\mathcal{M}$
and let their semantics be

\[
\begin{array}{rcl}
  \sembr{\bullet}^\ph_\mathcal{L} & : & \mathcal{L}\to\mathcal{D}\\
  \sembr{\bullet}^\ph_\mathcal{M} & : & \mathcal{M}\to\mathcal{D}
\end{array}
\]

We say that a transformation

\[
f : \mathcal{L}\to\mathcal{M}
\]

is \emph{semantically correct} if and only if for all programs $\primi{p}$

\[
\sembr{\primi{p}}^\ph_\mathcal{M}=\sembr{f\,(\primi{p})}^\ph_\mathcal{L}
\]

And, again, when $\mathcal{D}$ is a domain of functions the equation above denotes the equality of functions, and the notion of
partial correctness arises. One example of transformation between languages is compilation; as we already know, compilers as a rule
are partially correct.


\section{Interpreters, Compilers, Specializers}

We already mentioned compilation as a syntactic transformation from one language to another; we also
talked of compilers as programs which implement compilation. Here we consider them and some other useful
transformations in the form of programs in more details.

Let $\mathcal{L}$ and $\mathcal{M}$ be two languages, and

\[
\begin{array}{rcl}
\sembr{\bullet}^\ph_{\mathcal L} & : & \mathcal{L} \to \mathfrak{D} \to \mathfrak{D}\\
\sembr{\bullet}^\ph_{\mathcal M} & : & \mathcal{M} \to \mathfrak{D} \to \mathfrak{D}
\end{array}
\]

--- their semantics. An \emph{interpreter} for language $\mathcal{L}$, written in language $\mathcal{M}$, is a program $\Int{L}{M}\in\mathcal{M}$, such that for each
program $\primi{p}^\ph_\mathcal{L}\in\mathcal{L}$ and each data value $x\in\mathfrak{D}$

\[
\sembr{\Int{L}{M}}^\ph_\mathcal{M}\,(\primi{p}^\ph_\mathcal{L}\times x) = \sembr{\primi{p}^\ph_\mathcal{L}}^\ph_\mathcal{L}\,(x)\eqno{(\star)}
\]

To some extent an interpreter \emph{implements} the semantics of a programming language: given a program and its input it provides exactly the same result
this program calculates. Of course there can be many interpreters for a given pair of languages; any program $\primi{i}^\ph_\mathcal{M}$ satisfying equation $(\star)$,
e.g. such than

\[
\forall \primi{p}^\ph_\mathcal{L},\,x\in\mathfrak{D}\,.\,\sembr{\primi{i}^\ph_\mathcal{M}}^\ph_\mathcal{M}\,(\primi{p}^\ph_\mathcal{L}\times x)=\sembr{\primi{p}^\ph_\mathcal{L}}^\ph_\mathcal{L}\,(x)
\]

is an interpreter.

A particular interesting kind of interpreter is \emph{self}-interpreter $\Int{L}{L}$, i.e. an interpreter which interprets the language of its own implementation.
In the computability theory such interpreters are known under the name ``\emph{universal functions}'', and it is proven than universal functions exist for
all Turing-complete languages.

Another interesting program is, of course, compiler. Given languages $\mathcal{L}$, $\mathcal{M}$, and $\mathcal{N}$, a compiler from
$\mathcal L$ to $\mathcal N$, written in a language $\mathcal M$, is a program $\Comp{L}{N}{M}\in\mathcal M$ such that
for all programs $\primi{p}^\ph_\mathcal{L}\in\mathcal L$ and all input data values $x\in\mathfrak{D}$ the following equation holds:

\[
\sembr{\sembr{\Comp{L}{N}{M}}^\ph_\mathcal{M}\,(\primi{p}^\ph_\mathcal{L})}^\ph_\mathcal{N}\,(x) = \sembr{\primi{p}^\ph_\mathcal{L}}^\ph_\mathcal{L}\,(x)\eqno{(\star\star)}
\]

Indeed, a compiler takes a program representation $\primi{p}^\ph_\mathcal{L}$ as input and produces another program, $\sembr{\Comp{L}{N}{M}}^\ph_\mathcal{M}\,(\primi{p}^\ph_\mathcal{L})$,
this time in the language $\mathcal{N}$, which gives exactly the same result as $\primi{p}^\ph_\mathcal{L}$ for any input $x$. And, again, any program $\primi{c}^\ph_\mathcal{M}$ satisfying
the equation $(\star\star)$ is a compiler.

Finally, there can be a program called \emph{specializer} $\Spec{L}{M}$, written in a language $\mathcal M$ for a language $\mathcal L$, such that for
all programs $\primi{p}^\ph_\mathcal{L}\in\mathcal L$ and all data values $x, y\in\mathfrak D$

\[
\sembr{\sembr{\Spec{L}{M}}^\ph_\mathcal{M}\,(\primi{p}^\ph_\mathcal{L}\times x)}^\ph_\mathcal{L}\,(y)=\sembr{\primi{p}^\ph_\mathcal{L}}^\ph_\mathcal{L}\,(x\times y)\eqno{(\star\star\star)}
\]

Informally, a specializer takes as input a program $\primi{p}^\ph_\mathcal{L}$ and \emph{one} of its inputs $x$ and builds a program in the same language $\mathcal L$ which
takes $y$~--- the remaining inputs of $\primi{p}^\ph_\mathcal{L}$,~--- and provides exactly the same result as $\primi{p}^\ph_\mathcal{L}$ for both $x$ and $y$. The existence of specializers is,
again, guaranteed by the computability theory (\emph{Kleene $s$-$m$-$n$--theorem}).

%It's worth discussing why (and, actually, when) these programs exist. Obviously, if both $\mathcal{L}$ and $\mathcal{M}$ are Turing-complete, there exist
%both $\Int{L}{M}$ and $\Int{M}{L}$ (why?).

\section{Futamura Projections}

We now study an few elegant theoretical constructs which connect together the notions of interpreters, compilers and specializers. To simplify the presentation we
introduce the following denotation

\[
p^\ph_\mathcal{L}=\sembr{\primi{p}^\ph_\mathcal{L}}^\ph_\mathcal{L}
\]

for a program $\primi{p}^\ph_\mathcal{L}$. Thus, while $\primi{p}^\ph_\mathcal{L}$ is a program in a language $\mathcal{L}$ (i.e. a syntactic object), $p^\ph_\mathcal{L}$ is
its semantics (a function in the data domain).

Our first step is to apply a specializer $\Spec{L}{M}$ to some interpreter $\Int{N}{L}$ and some program $\primi{p}^\ph_\mathcal{N}$ it can interpret:

\[
\sembr{\underline{\SpecS{L}{M}\,(\Int{N}{L}\times \primi{p}^\ph_\mathcal{N})}}^\ph_\mathcal{L}\,(x)=\IntS{N}{L}\,(\primi{p}^\ph_\mathcal{N}\times x)=\sembr{\underline{\primi{p}^\ph_\mathcal{N}}}^\ph_\mathcal{N}\,(x)\eqno{(I)}
\]

The first equality follows immediately from $(\star\star\star)$ while the second~--- immediately from $(\star)$. Let's now look at the underlined parts. The \emph{right} one is, obviously,
$\primi{p}^\ph_\mathcal{N}$, a program in the language $\mathcal{N}$. The \emph{left} one is some program in the language $\mathcal{L}$. The equation itself states that the semantic of these
two programs give the same value for every input $x$. In other words, these two programs are equivalent. This is the first Futamura projection:

\begin{quote}
  \emph{The specialization of an interpreter for a program gives the representation of this program in the language of interpreter implementation.}
\end{quote}

Next, let's specialize a specializer for an interpreter:

\begin{multline*}
  \sembr{\sembr{\underline{\SpecS{M}{K}\,(\Spec{L}{M}\times\Int{N}{L})}}^\ph_\mathcal{K}\,(\primi{p}^\ph_\mathcal{N})}^\ph_\mathcal{L}\,(x)=\\
  \sembr{\SpecS{L}{M}\,(\Int{N}{L}\times \primi{p}^\ph_\mathcal{N})}^\ph_\mathcal{L}\,(x)=
  \sembr{\primi{p}^\ph_\mathcal{N}}^\ph_\mathcal{N}\,(x)\tag{II}
\end{multline*}

The first equation, again, immediately follows from $(\star\star\star)$, while the second~--- from $(I)$. If to look at the underlined part long enough, it becomes
evident that it is a program in the language $\mathcal{M}$ which satisfies the equation $(\star\star)$, i.e. a compiler $\Comp{N}{L}{M}$. This is a second Futamura projection:

\begin{quote}
  \emph{The specialization of a specializer to an interpreter gives a compiler from the interpreting language to the language of interpreter implementation.}
\end{quote}

Finally, we can specialize a specializer to a specializer:

\[
  \sembr{\SpecS{K}{T}\,(\Spec{M}{K}\times\Spec{L}{M})}^\ph_\mathcal{T}\,(\Int{N}{L})=\SpecS{M}{K}\,(\Spec{L}{M}\times\Int{N}{L})\eqno{(III)}
\]

The equation immediately follows from $(\star\star)$; its right part, according to $(II)$, is $\Comp{N}{L}{M}$. This is the
third Futamura projection:

\begin{quote}
  \emph{The specialization of a specializer to a specializer gives a compiler generator which for an interpreter generates a compiler from
  the interpreting language to the language of interpreter implementation.}
\end{quote}

Futamura projections are named after Y.Futamura, who described the first two of them in the beginning of 1970s. All three
projections were independently discovered by V.Turchin and A.Ershov, who gave them their current name.

The beauty of Futamura projections is that they give a rather simple equations for rather complex tools like compilers and compiler generators.
However, this immediately raises a question if one can indeed acquire these tools using such a high-level description.

Let's assume that we are going to use Futamura projections to implement a compiler from \lama to \texttt{x86}. Then we, first, need an interpreter
$\Int{\mbox{\lama}}{\mbox{\texttt{x86}}}$ for \lama written in \texttt{x86} assembler. The task of implementing such an interpreter, while involving
some low-level programming, does not look very challenging. Then, we need a specializer $\Spec{\mbox{\texttt{x86}}}{\mathcal{L}}$ for \texttt{x86}
assembler written in some language $\mathcal{L}$, not necessarily \lama. We can choose any suitable language for this purpose. Having both at
hands, we will be able to compile \lama-programs to \texttt{x86} code using the first Futamura projection:

\[
\SpecS{\mbox{\texttt{x86}}}{\mathcal{L}}\,(\Int{\mbox{\lama}}{\mbox{\texttt{x86}}}\times \primi{p}^\ph_{\mbox{\lama}})=\primi{p}^\ph_{\mbox{\texttt{x86}}}
\]


And here comes the hard part: the simplest possible specializer (for example, that guaranteed by the $s$-$m$-$n$--theorem) would produce a very
poor machine code; it would, in fact, just link the interpreter with the program, which invalidates the very idea of compilation. In order to
acquire a decent result, the specializer has to be non-trivial. The task of developing a non-trivial specializer even for the first Futamura
projection is non-trivial as well; nevertheless there are frameworks where this task is solved. For example, \texttt{GraalVM} uses the first Futamura projection
as a tool for language bootstrapping.

If we move higher in the Futamura projection hierarchy we would need at least one additional specializer $\Spec{L}{L}$, this time
for the language $\mathcal{L}$; it can be written in the $\mathcal{L}$ as well. This specializer has to be even more advanced than
$\Spec{\mbox{\texttt{x86}}}{\mathcal{L}}$ since we expect it to decently specialize more complicated program, than interpreters.

Finally, for the third Futamura projection we need even more advanced specializer since it has to be capable of decently specialize specializer for a
specializer. Note, is the third Futamura projections the last two specializers need not necessarily be the same programs, but it is very
appealing from both scientific and aesthetic standpoints to have the single, \emph{self-applicable}, specializer.

Thus, using Futamura projections beyond the first one in practice is still a hard venture. In the middle of 1980s all three projections were
implemented by the group led by N.Jones, but this was rather a proof-of-concept than a working industrial technology.
