\documentclass{article}

\usepackage[utf8]{inputenc}
\usepackage[english,russian]{babel}
\usepackage[T2A]{fontenc}

\begin{document}

\section{Языки программирования}

Наш курс посвящен компиляторам языков программирования, поэтому было бы разумно прежде всего обсудить, что такое, собственно,
языки программирования. В первом приближении можно сказать, что языки программирования~--- это языки для записи программ для компъютера.
Такое объяснение, при всей его тавтологичности, тем не менее содержит важное наблюдение: коль скоро языки программирования~--- это именно \emph{языки},
то есть знаковые системы, то для рассуждения о языках программирования следует применять понятийный аппарат \emph{семиотики} (науки о
знаковых системах).

Один из основателей этой науки, Чарльз Уильям Моррис (Charles William Morris), выделил три важных понятия:

\begin{itemize}
\item \emph{Синтаксис}~--- отношения знаков между собой.
\item \emph{Семантика}~--- отношения знакa к объекту.
\item \emph{Прагматика}~--- отношения знакa к субъекту.
\end{itemize}

В применении к языкам программирования синтаксис означает форму представления программ, семантика~--- их смысл, а прагматика~--- взаимодействие
языка программирования и программиста. Вопросы, связанные с прагматикой языков программирования, в нашем курсе рассматриваться не будут, а
вот синтаксис и семантика окажутся предметом особого внимания.

\section{Синтаксис}

Подобно естественным языкам, в синтаксисе языков программирования как правило можно выделить несколько уровней (лексика, грамматика и т.д.) Однако, в отличие
от естественных языков, которые развивались в значительной степени стихийно, языки программирования разрабатываются специально с
учетом определенных требований. В частности, синтаксис языков программирования \emph{однозначен} и приспособлен для эффективного автоматического анализа.

Для иллюстрации многоуровневого подхода с синтаксису рассмотрим следующий простой фрагмент на языке С. С точки зрения
\emph{лексики}, этот фрагмент является последовательностью \emph{лексем} (ключевое слово, разделитель, идентификатор, знак бинарной операции, десятичная константа и т.д.)
Эта последовательность лексем в свою очередь образует иерархию синтаксических конструкций (в данном случае, выражений и операторов).

Естественные языки как правило неоднозначны. Рассмотрим такую фразу на русском языке: ``Входя в двери лифта с животными, придерживайте их''. Придерживать кого? Двери или
животных? В данном случае мы имеем глобальную неоднозначность, которую невозможно разрешить даже путем анализа контекста. Единственный способ её устранить --- это изменить фразу
(``Входя в лифт с животными, придерживайте двери'' либо ``Придерживайте животных, входя с ними в двери лифта''). Наличие такого рода неразрешимых неоднозначностей в
языках программирования недопустимо.

\section{Семантика}

Наиболее ярко отличия языков программирования от естественных языков проявляются на уровне семантики. Для языков программирования существуют формальные
средства спецификации их семантики, которые позволяют получить доказуемо надежные результаты.

Почему важна формальная семантика? Если в большинстве случаев при обычном программировании мы опираемся на неформальные представления о смысле программ,
то при разработке инструментальных средств, в частности, компиляторов, таких представлений оказывается недостаточно. Представим, например, что вам сказали,
что в каком-то языке программирования выражения состоят из переменных, констант и знаков четырех арифметических действий. Достаточно ли этой информации для
того, чтобы написать компилятор?  

Для того, чтобы проиллюстрировать, как нас могут подвести интуитивные представления, рассмотрим пример небольшой программы на языке С. Язык С является одним из
наиболее распространённых в мире, он не считается ни трудным, ни ``эзотерическим'', поэтому можно ожидать, что любой человек, знакомый с этим языком, в
состоянии разобраться, что делает программа, состоящая из 12 строк. Вот эта программа.

Может показаться, что я вас обманываю, и что это не программа на языке С, а просто бессвязный набор символов. Проверим это. <демонстрация>

Как видим, для компилятора С эта совершенно непонятная для нас программа ничем не отличается от любой другой. Данная программа была разработана для конкурса по
``запутыванию программ'', больше подобных примеров можно найти на ресурсе https://www.cise.ufl.edu/~manuel/obfuscate/obfuscate.html. 

\section{Сущность трасляции}

Трансляция~--- это синтаксическое преобразование программ на одном языке в (эквивалентные) программы на другом. В возможности реализации автоматической
трансляции заключается важное отличие языков программирования от естественных языков. Несмотря на гигантский прогресс в области автоматического перевода
для естественных языков, качество этого перевода зачастую становится предметом дискуссий, и уж совершенно точно корректность трансляции для естественного
языка не может быть формально доказана (по крайней мере в обозримом будущем).



\end{document}

